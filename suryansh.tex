\documentclass[12pt,a4paper]{article}

\usepackage[margin=2.4cm,top=3cm,bottom=3cm]{geometry}
\usepackage{xcolor}
\usepackage{titlesec}
\usepackage{enumitem}
\usepackage{fancyhdr}
\usepackage{lastpage}
\usepackage{tabularx}
\usepackage{booktabs}
\usepackage{listings}
\usepackage{parskip}
\usepackage{amsmath}
\usepackage{tikz}
\usetikzlibrary{calc}
\usepackage{eso-pic}

% ────────────────────────────────────────
%   Code Styling (Improved & Clean)
% ────────────────────────────────────────
\definecolor{codebg}{RGB}{245, 247, 250}
\definecolor{codeblue}{RGB}{37, 99, 235}
\definecolor{codegreen}{RGB}{22, 163, 74}
\definecolor{codered}{RGB}{220, 38, 38}

\lstset{
    language=Python,
    backgroundcolor=\color{codebg},
    basicstyle=\ttfamily\small,
    keywordstyle=\color{codeblue}\bfseries,
    commentstyle=\color{codegreen},
    stringstyle=\color{codered},
    numbers=left,
    numberstyle=\tiny\color{gray},
    stepnumber=1,
    numbersep=8pt,
    breaklines=true,
    breakatwhitespace=true,
    frame=single,
    rulecolor=\color{gray!40},
    tabsize=4,
    showstringspaces=false
}

% Color Definitions
\definecolor{titleblue}{RGB}{30, 58, 138}
\definecolor{accent}{RGB}{59, 130, 246}
\definecolor{graydark}{RGB}{75, 85, 99}

% Section Formatting
\titleformat{\section}
  {\normalfont\Large\bfseries\color{titleblue}}
  {}
  {0em}
  {}

\titleformat{\subsection}
  {\normalfont\large\bfseries\color{accent}}
  {}
  {0em}
  {\makebox[0pt][l]{\color{gray!30}\rule[-0.35ex]{1.1cm}{2.2pt}}\enspace}

% % Header
% \pagestyle{fancy}
% \fancyhf{}
% \fancyhead[L]{\small\color{graydark} Data Analytics}
% \fancyhead[R]{\small\color{graydark} Page \thepage\ of \pageref{LastPage}}
% \renewcommand{\headrulewidth}{0.5pt}
% \renewcommand{\headrule}{\color{gray!20}\hrule}

% ────────────────────────────────────────
%   Footer Only (No Header)
% ────────────────────────────────────────
\pagestyle{fancy}
\fancyhf{}                        % Clear all header and footer fields
\fancyfoot[C]{\thepage}           % Page number centered in footer
\renewcommand{\headrulewidth}{0pt} % Remove header line
\renewcommand{\footrulewidth}{0pt} % No footer line

\setlength{\parindent}{0pt}
\setlength{\parskip}{0.85ex plus 0.4ex minus 0.3ex}
\setlist[enumerate]{leftmargin=*, label=\textbf{\arabic*.}}

% ────────────────────────────────────────
%   Double Page Border (Fixed Version)
% ────────────────────────────────────────
\usepackage{eso-pic}

\AddToShipoutPictureBG{
\begin{tikzpicture}[remember picture,overlay]

    % Outer Border
    \draw[black, line width=0.4pt]
        ($(current page.north west)+(0.6cm,-0.6cm)$)
        rectangle
        ($(current page.south east)+(-0.6cm,0.6cm)$);

    % Inner Border
    \draw[black, line width=0.3pt]
        ($(current page.north west)+(0.75cm,-0.75cm)$)
        rectangle
        ($(current page.south east)+(-0.75cm,0.75cm)$);

\end{tikzpicture}
}

% \AddToShipoutPictureBG{
% \begin{tikzpicture}[remember picture,overlay]

%     % Outer Border (near page edge)
%     \draw[line width=0.4pt]
%         ($(current page.north west)+(0.6cm,-0.6cm)$)
%         rectangle
%         ($(current page.south east)+(-0.6cm,0.6cm)$);

%     % Inner Border (slightly inside outer)
%     \draw[line width=0.3pt]
%         ($(current page.north west)+(0.75cm,-0.75cm)$)
%         rectangle
%         ($(current page.south east)+(-0.75cm,0.75cm)$);

% \end{tikzpicture}
% }

\begin{document}

% ──────────────────────────────
% Title Block
% ──────────────────────────────
\begin{center}
\vspace*{1.4cm}

{\LARGE\bfseries\color{titleblue} Data Analytics}\\[5mm]
{\Large\bfseries Assignment 1}\\[4mm]
{\large\color{accent} Subject Code: BCSDS0651 \quad$\vert$\quad Even Semester 2025-26}\\[1.2cm]

\begin{tabularx}{\textwidth}{>{\raggedleft\arraybackslash}p{3.4cm} >{\raggedright\arraybackslash}X}
\toprule[1pt]
\addlinespace[0.6em]
\textbf{\color{titleblue} Name} & Suryansh Singh \\
\textbf{\color{titleblue} Roll No.} & 2301330100208 \\
\textbf{\color{titleblue} Branch} & Computer Science and Engineering \\
\textbf{\color{titleblue} Semester} & 6 \\
\textbf{\color{titleblue} Section} & C \\
\textbf{\color{titleblue} Submitted to} & Ms. Ayushi Gupta \\
\addlinespace[0.6em]
\bottomrule[1pt]
\end{tabularx}
\end{center}

\newpage

% ──────────────────────────────
\section*{1. Write a Python program to compute Central Tendency and Dispersion Measures}

\subsection*{Mathematical Formulas}

Mean:
\[
\bar{x} = \frac{1}{n} \sum_{i=1}^{n} x_i
\]

Sample Variance:
\[
s^2 = \frac{1}{n-1} \sum_{i=1}^{n} (x_i - \bar{x})^2
\]

Standard Deviation:
\[
s = \sqrt{s^2}
\]

\subsection*{Python Program}

\begin{lstlisting}
import statistics

data = [10, 20, 20, 30, 40, 50]

# Central Tendency
mean_val = statistics.mean(data)
median_val = statistics.median(data)
mode_val = statistics.multimode(data)

# Dispersion (Sample)
variance_val = statistics.variance(data)
std_dev_val = statistics.stdev(data)

print("Mean:", mean_val)
print("Median:", median_val)
print("Mode:", mode_val)
print("Variance:", variance_val)
print("Standard Deviation:", std_dev_val)
\end{lstlisting}

% ──────────────────────────────
\section*{2. Study of Python Basic Libraries: Statistics, Math, NumPy and SciPy}

\subsection*{Statistics Library}
Provides built-in functions for mean, median, variance, and standard deviation.

\subsection*{Math Library}
Used for mathematical operations such as:
\begin{itemize}
\item \texttt{math.sqrt()}
\item \texttt{math.log()}
\item \texttt{math.factorial()}
\end{itemize}

\subsection*{NumPy}
Efficient numerical computations using arrays.

\begin{lstlisting}
import numpy as np

arr = np.array([10, 20, 20, 30, 40, 50])

print("NumPy Mean:", np.mean(arr))
print("NumPy Variance:", np.var(arr))
\end{lstlisting}

\subsection*{SciPy}
Used for advanced statistical functions.

\begin{lstlisting}
from scipy import stats

print("SciPy Mode:", stats.mode(arr, keepdims=True))
\end{lstlisting}

% ──────────────────────────────
\section*{3. Study of Python Libraries for ML Applications: Pandas and Matplotlib}

\subsection*{Pandas}
Used for structured data manipulation.

\begin{lstlisting}
import pandas as pd

df = pd.DataFrame({"Values": arr})
print(df.describe())
\end{lstlisting}

\subsection*{Matplotlib}
Used for visualization.

\begin{lstlisting}
import matplotlib.pyplot as plt

plt.hist(arr)
plt.title("Histogram of Data")
plt.xlabel("Values")
plt.ylabel("Frequency")
plt.show()
\end{lstlisting}

% ──────────────────────────────
\section*{4. Write a Python program to implement Simple Linear Regression}

\subsection*{Mathematical Model}

\[
y = \beta_0 + \beta_1 x
\]

\[
\beta_1 =
\frac{\sum (x_i - \bar{x})(y_i - \bar{y})}
{\sum (x_i - \bar{x})^2}
\]

\[
\beta_0 = \bar{y} - \beta_1 \bar{x}
\]

\subsection*{Python Program}

\begin{lstlisting}
import numpy as np
import matplotlib.pyplot as plt

x = np.array([1, 2, 3, 4, 5])
y = np.array([2, 4, 5, 4, 5])

x_mean = np.mean(x)
y_mean = np.mean(y)

beta_1 = np.sum((x - x_mean) * (y - y_mean)) \
         / np.sum((x - x_mean)**2)

beta_0 = y_mean - beta_1 * x_mean

print("Intercept:", beta_0)
print("Slope:", beta_1)

y_pred = beta_0 + beta_1 * x

plt.scatter(x, y)
plt.plot(x, y_pred)
plt.title("Simple Linear Regression")
plt.xlabel("X")
plt.ylabel("Y")
plt.show()
\end{lstlisting}

% ──────────────────────────────
\section*{5. Load a Sample Dataset to Build a Predictive Model in Python}

\subsection*{Problem Statement}
\begin{enumerate}
\item Import Python Libraries
\item Read the Dataset
\end{enumerate}

\subsection*{Solution}

\begin{lstlisting}
# 1. Import Required Libraries
import pandas as pd
import numpy as np
from sklearn.datasets import load_iris

# 2. Load Sample Dataset (Iris Dataset)
iris = load_iris()

# Convert to DataFrame
df = pd.DataFrame(data=iris.data, columns=iris.feature_names)
df['target'] = iris.target

print(df.head())
\end{lstlisting}

% ──────────────────────────────
\section*{6. Explore the Dataset and Perform Feature Selection}

\subsection*{Problem Statement}
\begin{enumerate}
\item Explore the Dataset
\item Perform Feature Selection
\end{enumerate}

\subsection*{Solution}

\begin{lstlisting}
# Dataset Exploration
print("Dataset Shape:", df.shape)
print(df.info())
print(df.describe())

# Feature Selection
X = df.drop("target", axis=1)
y = df["target"]

print("Selected Features:")
print(X.columns)
\end{lstlisting}

% ──────────────────────────────
\section*{7. Build the Model and Evaluate Performance}

\subsection*{Problem Statement}
\begin{enumerate}
\item Build the Model
\item Evaluate the Model's Performance
\end{enumerate}

\subsection*{Solution}

\begin{lstlisting}
from sklearn.model_selection import train_test_split
from sklearn.linear_model import LogisticRegression
from sklearn.metrics import accuracy_score, classification_report

# Split Data
X_train, X_test, y_train, y_test = train_test_split(
    X, y, test_size=0.3, random_state=42
)

# Build Model
model = LogisticRegression(max_iter=200)
model.fit(X_train, y_train)

# Predictions
y_pred = model.predict(X_test)

# Evaluation
print("Accuracy:", accuracy_score(y_test, y_pred))
print("Classification Report:\n", classification_report(y_test, y_pred))
\end{lstlisting}

% ──────────────────────────────
\section*{8. Installing and Setting Up Python Environment}

\subsection*{Problem Statement}
\begin{enumerate}
\item Installing Pandas and Other Dependent Python Modules
\item Setting Up and Using Jupyter Notebooks
\end{enumerate}

\subsection*{Solution}

\textbf{Step 1: Install Python}

Download Python from the official website and verify installation:

\begin{lstlisting}
python --version
\end{lstlisting}

\textbf{Step 2: Install Required Libraries}

\begin{lstlisting}
pip install numpy pandas matplotlib scipy scikit-learn
\end{lstlisting}

\textbf{Step 3: Install Jupyter Notebook}

\begin{lstlisting}
pip install notebook
\end{lstlisting}

\textbf{Step 4: Launch Jupyter Notebook}

\begin{lstlisting}
jupyter notebook
\end{lstlisting}

Jupyter Notebook opens in a web browser where Python code can be executed interactively.

\end{document}